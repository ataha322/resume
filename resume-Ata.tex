%-------------------------
% Resume in Latex
% Author : Sidratul Muntaha Ahmed
% License : MIT
%------------------------

\documentclass[letterpaper,11pt]{article}

\usepackage{latexsym}
\usepackage[empty]{fullpage}
\usepackage{titlesec}
\usepackage{marvosym}
\usepackage[usenames,dvipsnames]{color}
\usepackage{verbatim}
\usepackage{enumitem}
\usepackage[hidelinks]{hyperref}
\usepackage{fancyhdr}
\usepackage[english]{babel}
\usepackage{tabularx}
\usepackage{amsmath}
\input{glyphtounicode}


%----------FONT OPTIONS----------
% sans-serif
% \usepackage[sfdefault]{FiraSans}
% \usepackage[sfdefault]{roboto}
% \usepackage[sfdefault]{noto-sans}
% \usepackage[default]{sourcesanspro}

% serif
% \usepackage{CormorantGaramond}
% \usepackage{charter}


\pagestyle{fancy}
\fancyhf{} % clear all header and footer fields
\fancyfoot{}
\renewcommand{\headrulewidth}{0pt}
\renewcommand{\footrulewidth}{0pt}

% Adjust margins
\addtolength{\oddsidemargin}{-0.5in}
\addtolength{\evensidemargin}{-0.5in}
\addtolength{\textwidth}{1in}
\addtolength{\topmargin}{-.5in}
\addtolength{\textheight}{1.0in}

\urlstyle{same}

\raggedbottom
\raggedright
\setlength{\tabcolsep}{0in}

% Sections formatting
\titleformat{\section}{
  \vspace{-4pt}\scshape\raggedright\large
}{}{0em}{}[\color{black}\titlerule \vspace{-5pt}]

% Ensure that generate pdf is machine readable/ATS parsable
\pdfgentounicode=1

%-------------------------
% Custom commands
\newcommand{\resumeItem}[1]{
  \item\small{
    {#1 \vspace{-2pt}}
  }
}

\newcommand{\resumeSubheading}[4]{
  \vspace{-2pt}\item
    \begin{tabular*}{0.97\textwidth}[t]{l@{\extracolsep{\fill}}r}
      \textbf{#1} & #2 \\
      \textit{\small#3} & \textit{\small #4} \\
    \end{tabular*}\vspace{-7pt}
}

\newcommand{\resumeSubSubheading}[2]{
    \item
    \begin{tabular*}{0.97\textwidth}{l@{\extracolsep{\fill}}r}
      \textit{\small#1} & \textit{\small #2} \\
    \end{tabular*}\vspace{-7pt}
}

\newcommand{\resumeProjectHeading}[2]{
    \item
    \begin{tabular*}{0.97\textwidth}{l@{\extracolsep{\fill}}r}
      \small#1 & #2 \\
    \end{tabular*}\vspace{-7pt}
}

\newcommand{\resumeSubItem}[1]{\resumeItem{#1}\vspace{-4pt}}

\renewcommand\labelitemii{$\vcenter{\hbox{\tiny$\bullet$}}$}

\newcommand{\resumeSubHeadingListStart}{\begin{itemize}[leftmargin=0.15in, label={}]}
\newcommand{\resumeSubHeadingListEnd}{\end{itemize}}
\newcommand{\resumeItemListStart}{\begin{itemize}}
\newcommand{\resumeItemListEnd}{\end{itemize}\vspace{-5pt}}

%-------------------------------------------
%%%%%%  RESUME STARTS HERE  %%%%%%%%%%%%%%%%%%%%%%%%%%%%


\begin{document}

%----------HEADING----------


\begin{center}
    \textbf{\Huge \scshape Ata Altyyev} \\ \vspace{1pt}
    \small 209-446-7533 $|$ \href{mailto:altyew44@gmail.com}{\underline{altyew44@gmail.com}} $|$ 
    \href{https://github.com/ataha322}{\underline{github.com/ataha322}}
\end{center}


%-----------EDUCATION-----------
\section{Education}
  \resumeSubHeadingListStart
    \resumeSubheading
      {University of California, San Diego}{La Jolla, CA}
      {BS, Mathematics and Computer Science, GPA: 3.6}{Expected January 2024}
          \resumeItemListStart
            \resumeItem{\textbf{Relevant Coursework}}:\\
            \qquad Advanced Data Structures and Algorithms, Systems Programming (ARMv8 Assembly), Digital Systems/RTL design, Theory of Computations (Finite Automatons and Turing Machines), Algebraic Combinatorics.
            \resumeItem{\textbf{Self-taught}}:\\
            \qquad Kernighan \& Ritchie, The C Programming Language\\
            \qquad Anderson and Dahlin, Operating Systems - Principles $\&$ Practice
        \resumeItemListEnd

  \resumeSubHeadingListEnd

%-----------EXPERIENCE-----------
\section{Work Experience}
  \resumeSubHeadingListStart

    \resumeSubheading
      {University Research - Embedded Systems \& Tiny ML}{September 2022 – Present}
      {University of California, San Diego}{San Diego, California}
      \resumeItemListStart
        \resumeItem{The research in ultra-low power embedded systems of a very small area that perform edge computations in ML models of size of several kilobytes.}
        \resumeItem{The main challenge is to accommodate the resources of a microcontroller for the model to run more effectively. This includes: hardware acceleration, network systems, benchmarking, model compression, learning algorithms.}
      \resumeItemListEnd
     
    \resumeSubHeadingListEnd

%-----------PROJECTS-----------
\section{Projects}

    \resumeSubHeadingListStart

    \resumeProjectHeading
          {\textbf{Rendering} $|$ \emph{Real time rendered graphics}}{August 2022}
          
          \resumeItemListStart
            \resumeItem{Real time rendered graphics using C++, SFML, OpenGL.}
            \resumeItem{This is a part of my learning of OpenGL, guided project. Implemented a moving camera, randomly generated buildings, fog, built a texture cube.}
            \resumeItem{Technical significancies lied in bitwise operations and geometry calculations.}
            \resumeItem{\url{https://github.com/ataha322/opengl-render-city}}
          \resumeItemListEnd
          
    \resumeProjectHeading
          {\textbf{Newton's box} $|$ \emph{2D Gravity simulation}}{July 2022}
          \resumeItemListStart
            \resumeItem{2D planet gravity simulation. Moon rotates around its planet where planet is a movable object to demostrate changes in inertial and accelerated frames.}
            \resumeItem{Used C++ and SFML library to implement two key objects: planet and its moon. Gravity calculations are made in the moon object, with planet object passed in. Planet is movable, simulation is resettable, window frames are adjoint.}
            \resumeItem{\url{https://github.com/ataha322/newtonBox}}
          \resumeItemListEnd         

    \resumeProjectHeading
          {\textbf{Planner.xyi} $|$ \emph{Web-application}}{June 2022 -- Aug 2022}
          \resumeItemListStart
            \resumeItem{Planner/Calendar/Notepad application. Initially implemented as a web app but will be ported on android. The structure is simple: User-Task interaction. Task modules communicate with user modules through binded UserId's, which allows to store multiple users with their private tasks. Features implemented: registration, login, sort and search, deadline counting, email verification, authentication.}
            \resumeItem{Wrote the backing code with Golang due to the use of the GORM library and use of concurrency with goroutines.}
            \resumeItem{Packaged this program into the docker container for its easy portability.}
            \resumeItem{Stored data in MySQL tables. Cached and encrypted the data with Redis and JWT respectively.}
            \resumeItem{Frontend was implemented with the use of VueJS, Nuxt.js, and Vuetify.}
            \resumeItem{Group Project: backend - \textit{Ata Altyyev}(me), frontend - \href{https://github.com/dzodkin33}{\textit{Boris Ryabov}}.\\
            \url{https://github.com/ataha322/planner.xyi}
            \url{https://github.com/dzodkin33/planner-front}}
          \resumeItemListEnd 

    \resumeSubHeadingListEnd



%
%-----------TECHNICAL SKILLS-----------
\section{Technical Skills}
 \begin{itemize}[leftmargin=0.15in, label={}]
    \small{\item{
     \textbf{Languages}{: C/C++, ARM Assembly, Golang, Java, Python, Pascal} \\
     \textbf{Libraries \& API}{: Redis, Gorm(MySQL), OpenGL, SFML, Fiber, JWT, Stripe} \\
     \textbf{Developer Tools}{: Docker, GDB, Valgrind, Linux, Git, bash \& make scripts, RaspberryPi (C-code, ARM-code), \LaTeX} \\
     \textbf{Skills}{: Object-Oriented Programming, ASM reverse engineering, Test-Driven Development, Matlab/Numpy} \\
     \textbf{Miscellaneous}{: Burnt serial programmer by connecting two power sources, \href{https://github.com/ataha322/dvd-bounce}{{\underline{DVD-like bouncing screensaver}}}}
    }}
 \end{itemize}
 


%-------------------------------------------
\end{document}